\documentclass{article}
\usepackage[utf8]{inputenc}
\usepackage[brazil]{babel}
\usepackage{setspace}
\usepackage{mathtools}
\usepackage{pgfplots}
\usepackage{listings}
\usepackage{xcolor}
\usepackage{natbib}
\usepackage{graphicx}

\DeclarePairedDelimiter\ceil{\lceil}{\rceil}
\DeclarePairedDelimiter\floor{\lfloor}{\rfloor}

\definecolor{codegreen}{rgb}{0,0.6,0}
\definecolor{codegray}{rgb}{0.5,0.5,0.5}
\definecolor{codepurple}{rgb}{0.58,0,0.82}
\definecolor{backcolour}{rgb}{0.95,0.95,0.92}

\lstdefinestyle{codigo}{
    numberstyle=\tiny,
    basicstyle=\ttfamily\footnotesize,
    breakatwhitespace=false,         
    breaklines=true,                 
    captionpos=b,                    
    keepspaces=true,                 
    numbers=left,                    
    numbersep=5pt,                  
    showspaces=false,                
    showstringspaces=false,
    showtabs=false,                  
    tabsize=2
}

\lstset{style=codigo}

\setstretch{1.5}
\pgfplotsset{width=10cm,compat=1.9}

\pagenumbering{gobble}
\clearpage
\thispagestyle{empty}

\title{Implementing a graph in C++}
\author{Lucas Santiago \and Rafael Amauri}
\date{March, 2021}



\begin{document}
\maketitle

\hspace{4pt}We implemented a graph thinking about the performance of insertions and removal.
Letting the overhead to the print function. We thought about how to make a very generic object
to simplify our future use, I'll explain better in the nexts paragraphs.

$Graph$ class is the the lowest-level implementation, this class handles 
everything in the basic-level. Inserting, removing and printing on the screen the values.

The $Vertex$ and $Edge$ classes are made to hold the graph values and they got connected in the $Graph$.
Both classes have their own values only.


\end{document}